% Options for packages loaded elsewhere
\PassOptionsToPackage{unicode}{hyperref}
\PassOptionsToPackage{hyphens}{url}
\PassOptionsToPackage{dvipsnames,svgnames,x11names}{xcolor}
%
\documentclass[
  11pt,
]{article}

\usepackage{amsmath,amssymb}
\usepackage{iftex}
\ifPDFTeX
  \usepackage[T1]{fontenc}
  \usepackage[utf8]{inputenc}
  \usepackage{textcomp} % provide euro and other symbols
\else % if luatex or xetex
  \usepackage{unicode-math}
  \defaultfontfeatures{Scale=MatchLowercase}
  \defaultfontfeatures[\rmfamily]{Ligatures=TeX,Scale=1}
\fi
\usepackage{lmodern}
\ifPDFTeX\else  
    % xetex/luatex font selection
\fi
% Use upquote if available, for straight quotes in verbatim environments
\IfFileExists{upquote.sty}{\usepackage{upquote}}{}
\IfFileExists{microtype.sty}{% use microtype if available
  \usepackage[]{microtype}
  \UseMicrotypeSet[protrusion]{basicmath} % disable protrusion for tt fonts
}{}
\makeatletter
\@ifundefined{KOMAClassName}{% if non-KOMA class
  \IfFileExists{parskip.sty}{%
    \usepackage{parskip}
  }{% else
    \setlength{\parindent}{0pt}
    \setlength{\parskip}{6pt plus 2pt minus 1pt}}
}{% if KOMA class
  \KOMAoptions{parskip=half}}
\makeatother
\usepackage{xcolor}
\usepackage[margin=1in]{geometry}
\setlength{\emergencystretch}{3em} % prevent overfull lines
\setcounter{secnumdepth}{-\maxdimen} % remove section numbering
% Make \paragraph and \subparagraph free-standing
\ifx\paragraph\undefined\else
  \let\oldparagraph\paragraph
  \renewcommand{\paragraph}[1]{\oldparagraph{#1}\mbox{}}
\fi
\ifx\subparagraph\undefined\else
  \let\oldsubparagraph\subparagraph
  \renewcommand{\subparagraph}[1]{\oldsubparagraph{#1}\mbox{}}
\fi

\usepackage{color}
\usepackage{fancyvrb}
\newcommand{\VerbBar}{|}
\newcommand{\VERB}{\Verb[commandchars=\\\{\}]}
\DefineVerbatimEnvironment{Highlighting}{Verbatim}{commandchars=\\\{\}}
% Add ',fontsize=\small' for more characters per line
\usepackage{framed}
\definecolor{shadecolor}{RGB}{241,243,245}
\newenvironment{Shaded}{\begin{snugshade}}{\end{snugshade}}
\newcommand{\AlertTok}[1]{\textcolor[rgb]{0.68,0.00,0.00}{#1}}
\newcommand{\AnnotationTok}[1]{\textcolor[rgb]{0.37,0.37,0.37}{#1}}
\newcommand{\AttributeTok}[1]{\textcolor[rgb]{0.40,0.45,0.13}{#1}}
\newcommand{\BaseNTok}[1]{\textcolor[rgb]{0.68,0.00,0.00}{#1}}
\newcommand{\BuiltInTok}[1]{\textcolor[rgb]{0.00,0.23,0.31}{#1}}
\newcommand{\CharTok}[1]{\textcolor[rgb]{0.13,0.47,0.30}{#1}}
\newcommand{\CommentTok}[1]{\textcolor[rgb]{0.37,0.37,0.37}{#1}}
\newcommand{\CommentVarTok}[1]{\textcolor[rgb]{0.37,0.37,0.37}{\textit{#1}}}
\newcommand{\ConstantTok}[1]{\textcolor[rgb]{0.56,0.35,0.01}{#1}}
\newcommand{\ControlFlowTok}[1]{\textcolor[rgb]{0.00,0.23,0.31}{#1}}
\newcommand{\DataTypeTok}[1]{\textcolor[rgb]{0.68,0.00,0.00}{#1}}
\newcommand{\DecValTok}[1]{\textcolor[rgb]{0.68,0.00,0.00}{#1}}
\newcommand{\DocumentationTok}[1]{\textcolor[rgb]{0.37,0.37,0.37}{\textit{#1}}}
\newcommand{\ErrorTok}[1]{\textcolor[rgb]{0.68,0.00,0.00}{#1}}
\newcommand{\ExtensionTok}[1]{\textcolor[rgb]{0.00,0.23,0.31}{#1}}
\newcommand{\FloatTok}[1]{\textcolor[rgb]{0.68,0.00,0.00}{#1}}
\newcommand{\FunctionTok}[1]{\textcolor[rgb]{0.28,0.35,0.67}{#1}}
\newcommand{\ImportTok}[1]{\textcolor[rgb]{0.00,0.46,0.62}{#1}}
\newcommand{\InformationTok}[1]{\textcolor[rgb]{0.37,0.37,0.37}{#1}}
\newcommand{\KeywordTok}[1]{\textcolor[rgb]{0.00,0.23,0.31}{#1}}
\newcommand{\NormalTok}[1]{\textcolor[rgb]{0.00,0.23,0.31}{#1}}
\newcommand{\OperatorTok}[1]{\textcolor[rgb]{0.37,0.37,0.37}{#1}}
\newcommand{\OtherTok}[1]{\textcolor[rgb]{0.00,0.23,0.31}{#1}}
\newcommand{\PreprocessorTok}[1]{\textcolor[rgb]{0.68,0.00,0.00}{#1}}
\newcommand{\RegionMarkerTok}[1]{\textcolor[rgb]{0.00,0.23,0.31}{#1}}
\newcommand{\SpecialCharTok}[1]{\textcolor[rgb]{0.37,0.37,0.37}{#1}}
\newcommand{\SpecialStringTok}[1]{\textcolor[rgb]{0.13,0.47,0.30}{#1}}
\newcommand{\StringTok}[1]{\textcolor[rgb]{0.13,0.47,0.30}{#1}}
\newcommand{\VariableTok}[1]{\textcolor[rgb]{0.07,0.07,0.07}{#1}}
\newcommand{\VerbatimStringTok}[1]{\textcolor[rgb]{0.13,0.47,0.30}{#1}}
\newcommand{\WarningTok}[1]{\textcolor[rgb]{0.37,0.37,0.37}{\textit{#1}}}

\providecommand{\tightlist}{%
  \setlength{\itemsep}{0pt}\setlength{\parskip}{0pt}}\usepackage{longtable,booktabs,array}
\usepackage{calc} % for calculating minipage widths
% Correct order of tables after \paragraph or \subparagraph
\usepackage{etoolbox}
\makeatletter
\patchcmd\longtable{\par}{\if@noskipsec\mbox{}\fi\par}{}{}
\makeatother
% Allow footnotes in longtable head/foot
\IfFileExists{footnotehyper.sty}{\usepackage{footnotehyper}}{\usepackage{footnote}}
\makesavenoteenv{longtable}
\usepackage{graphicx}
\makeatletter
\def\maxwidth{\ifdim\Gin@nat@width>\linewidth\linewidth\else\Gin@nat@width\fi}
\def\maxheight{\ifdim\Gin@nat@height>\textheight\textheight\else\Gin@nat@height\fi}
\makeatother
% Scale images if necessary, so that they will not overflow the page
% margins by default, and it is still possible to overwrite the defaults
% using explicit options in \includegraphics[width, height, ...]{}
\setkeys{Gin}{width=\maxwidth,height=\maxheight,keepaspectratio}
% Set default figure placement to htbp
\makeatletter
\def\fps@figure{htbp}
\makeatother

\makeatletter
\makeatother
\makeatletter
\makeatother
\makeatletter
\@ifpackageloaded{caption}{}{\usepackage{caption}}
\AtBeginDocument{%
\ifdefined\contentsname
  \renewcommand*\contentsname{Table of contents}
\else
  \newcommand\contentsname{Table of contents}
\fi
\ifdefined\listfigurename
  \renewcommand*\listfigurename{List of Figures}
\else
  \newcommand\listfigurename{List of Figures}
\fi
\ifdefined\listtablename
  \renewcommand*\listtablename{List of Tables}
\else
  \newcommand\listtablename{List of Tables}
\fi
\ifdefined\figurename
  \renewcommand*\figurename{Figure}
\else
  \newcommand\figurename{Figure}
\fi
\ifdefined\tablename
  \renewcommand*\tablename{Table}
\else
  \newcommand\tablename{Table}
\fi
}
\@ifpackageloaded{float}{}{\usepackage{float}}
\floatstyle{ruled}
\@ifundefined{c@chapter}{\newfloat{codelisting}{h}{lop}}{\newfloat{codelisting}{h}{lop}[chapter]}
\floatname{codelisting}{Listing}
\newcommand*\listoflistings{\listof{codelisting}{List of Listings}}
\makeatother
\makeatletter
\@ifpackageloaded{caption}{}{\usepackage{caption}}
\@ifpackageloaded{subcaption}{}{\usepackage{subcaption}}
\makeatother
\makeatletter
\@ifpackageloaded{tcolorbox}{}{\usepackage[skins,breakable]{tcolorbox}}
\makeatother
\makeatletter
\@ifundefined{shadecolor}{\definecolor{shadecolor}{rgb}{.97, .97, .97}}
\makeatother
\makeatletter
\makeatother
\makeatletter
\makeatother
\ifLuaTeX
  \usepackage{selnolig}  % disable illegal ligatures
\fi
\IfFileExists{bookmark.sty}{\usepackage{bookmark}}{\usepackage{hyperref}}
\IfFileExists{xurl.sty}{\usepackage{xurl}}{} % add URL line breaks if available
\urlstyle{same} % disable monospaced font for URLs
\hypersetup{
  pdftitle={Peter Maximilian Fortunato},
  colorlinks=true,
  linkcolor={blue},
  filecolor={Maroon},
  citecolor={Blue},
  urlcolor={blue},
  pdfcreator={LaTeX via pandoc}}

\title{Peter Maximilian Fortunato}
\author{}
\date{}

\begin{document}
\maketitle
\ifdefined\Shaded\renewenvironment{Shaded}{\begin{tcolorbox}[enhanced, breakable, interior hidden, sharp corners, frame hidden, borderline west={3pt}{0pt}{shadecolor}, boxrule=0pt]}{\end{tcolorbox}}\fi

\hypertarget{contact}{%
\section{Contact}\label{contact}}

\begin{itemize}
\tightlist
\item
  📧 peter@solplots.com\\
\item
  🌐 \href{https://gira-solo.github.io/solplots-website}{solplots.com}\\
\item
  💼 \href{https://www.linkedin.com/in/peter-m-f/}{LinkedIn}\\
\item
  🐙 \href{https://github.com/gira-solo}{GitHub}
\end{itemize}

\begin{center}\rule{0.5\linewidth}{0.5pt}\end{center}

\hypertarget{summary}{%
\section{Summary}\label{summary}}

Peter is a data scientist with over three years of professional
experience and is currently a member of the Strategic Routing team at
Worldpay (formerly part of FIS). He is an expert in the R and SQL
programming languages as well as analytical storytelling.

\begin{center}\rule{0.5\linewidth}{0.5pt}\end{center}

\hypertarget{skills}{%
\section{Skills}\label{skills}}

\begin{itemize}
\tightlist
\item
  \textbf{Languages:} R, SQL, Python
\item
  \textbf{Tools:} Databricks, RStudio, Quarto, Git, Snowflake, Jupyter,
  Streetlight, ArcGIS,\\
\item
  \textbf{Libraries:} tidyverse (ggplot2, dplyr, etc.), Shiny, RMarkdown
\end{itemize}

\begin{center}\rule{0.5\linewidth}{0.5pt}\end{center}

\hypertarget{experience}{%
\section{Experience}\label{experience}}

\hypertarget{data-scientist-i-strategic-routing-worldpay-inc}{%
\subsection{Data Scientist I, Strategic Routing --- Worldpay,
Inc}\label{data-scientist-i-strategic-routing-worldpay-inc}}

\emph{Remote \textbar{} Dec 2023 -- Present}

\begin{itemize}
\tightlist
\item
  Implements network routing product to maximize merchant cost savings
  on card processing fees
\item
  Performs ad-hoc analyses simulating potential merchant savings using a
  ``least cost routing'' method
\item
  Engineers routing table update processes by integrating various
  documents and requests
\item
  Monitors product performance to match negotiated transaction volumes
  from network deals
\end{itemize}

\hypertarget{data-scientist-transit-rail-planning-vanasse-hangen-brustlin-inc.-vhb}{%
\subsection{Data Scientist, Transit \& Rail Planning --- Vanasse Hangen
Brustlin,
Inc.~(VHB)}\label{data-scientist-transit-rail-planning-vanasse-hangen-brustlin-inc.-vhb}}

\emph{Washington, D.C. \textbar{} Sept 2022 - Dec 2023}

\begin{itemize}
\tightlist
\item
  Analyzed General Transit Feed Specification (GTFS) data in conjunction
  with Automated Passenger Count data to program R scripts and output
  load factor and schedule adherence KPI for transit studies
\item
  Developed and maintains a suite of transit planning and
  origin/destination traffic analysis Shiny Apps known as Data Trippers
  that resulted in assisting clients' decision making processes
\end{itemize}

\hypertarget{data-scientist-ohio-kentucky-indiana-regional-council-of-governments-oki}{%
\subsection{Data Scientist --- Ohio Kentucky Indiana Regional Council of
Governments
(OKI)}\label{data-scientist-ohio-kentucky-indiana-regional-council-of-governments-oki}}

\emph{Cincinnati, Ohio \textbar{} May 2021 - Aug 2022}

\begin{itemize}
\tightlist
\item
  Repurposed open-source R scripts to develop inaugural method
  evaluating roadway safety performance in the 8-county OKI region which
  resulted in the Excess Expected Crashes (EEC) KPI
\item
  Presented methodology at area analytics conferences and the useR!
  Conference 2022 (recognized)
\item
  Visualized of existing conditions and commodity flow using Freight
  Analysis Framework (FAF)
\end{itemize}

\begin{center}\rule{0.5\linewidth}{0.5pt}\end{center}

\hypertarget{education}{%
\section{Education}\label{education}}

\textbf{M.S. in Business Analytics}\\
Miami University of Ohio, 2021

\textbf{B.A. in International Studies}\\
Miami University of Ohio, 2020

\begin{center}\rule{0.5\linewidth}{0.5pt}\end{center}

\hypertarget{projects}{%
\section{Projects}\label{projects}}

\begin{itemize}
\item
  \textbf{Safety Performance Evaluation of OKI's Regional Highway
  Network:} As part of my work in developing a process to evaluate
  roadway safety while working for OKI, I organized my research into a
  poster presentation and submitted it to the
  \href{https://user2022.r-project.org/program/posters/}{useR! 2022
  Conference}. The poster was recognized as one of five ``best of the
  conference''.
  \href{../assets/pdfs/Roadway\%20Safety\%20Poster\%20V2.pdf}{View PDF
  Report}
\item
  \textbf{Mediterranean Corridor Volunteer Work:} Supported Commissioner
  Josep Vicent Boira Maiques of Valencia, Spain to develop
  visualizations that examined existing rail infrastructures related to
  type of service and technical characteristics.\\
  \href{../assets/pdfs/Carta\%20Peter\%20Fortunato\%20v2.\%201.pdf}{Certification
  Letter (in Spanish)}
\end{itemize}

\begin{center}\rule{0.5\linewidth}{0.5pt}\end{center}

\hypertarget{exporting-to-pdf}{%
\subsection{📄 Exporting to PDF}\label{exporting-to-pdf}}

To render a PDF version:

\begin{Shaded}
\begin{Highlighting}[]
\ExtensionTok{quarto}\NormalTok{ render resume.qmd }\AttributeTok{{-}{-}to}\NormalTok{ pdf}
\end{Highlighting}
\end{Shaded}




\end{document}
